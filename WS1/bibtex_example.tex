\documentclass[11pt,letterpaper]{article}

% Load some basic packages that are useful to have
% and that should be part of any LaTeX installation.
%
% be able to include figures
\usepackage{graphicx}
% get nice colors
\usepackage{xcolor}

% change default font to Palatino (looks nicer!)
\usepackage[latin1]{inputenc}
\usepackage{mathpazo}
\usepackage[T1]{fontenc}
% load some useful math symbols/fonts
\usepackage{latexsym,amsfonts,amsmath,amssymb}

% comfort package to easily set margins
\usepackage[top=1in, bottom=1in, left=1in, right=1in]{geometry}

% control some spacings
%
% spacing after a paragraph
\setlength{\parskip}{.15cm}
% indentation at the top of a new paragraph
\setlength{\parindent}{0.0cm}

% some definitions for journal names
\def\aj{Astron. J.}
\def\apj{Astrophys. J.}
\def\apjl{Astrophys. J. Lett.}
\def\apjs{Astrophys. J. Supp. Ser. }
\def\aa{Astron. Astrophys. }
\def\aap{Astron. Astrophys. }
\def\araa{Ann.\ Rev. Astron. Astroph. }
\def\physrep{Phys. Rep. }
\def\mnras{Mon. Not. Roy. Astron. Soc. }
\def\prl{Phys. Rev. Lett.}
\def\prd{Phys. Rev. D.}
\def\apss{Astrophys. Space Sci.}
\def\cqg{Class. Quantum Grav.}



\begin{document}

\begin{center}
\Large
Ay190 -- Worksheet 01\\
Chatarin (Mee) Wong-u-railertkun\\
Date: \today
\end{center}

Today, we are making reference to O'Connor \& Ott's paper on the
progenitor dependence of the pre-explosion neutrino emission in
core-collapse supernovae~\cite{oconnor:13}.

Another paper comes from a program called Robo-AO, the project I worked with the past summer. 
The paper talks about using the Robo-AO system to take images of several Kepler exoplanet candidates
~\cite{law:13}.


\bibliographystyle{unsrt}
\bibliography{bibtex_example}

\end{document}

